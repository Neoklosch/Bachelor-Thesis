\chapter{Zusammenfassung und Ausblick}
Ziel dieser Arbeit war es, einen Prototypen zu entwickeln, der die Möglichkeiten des Dateimanagements unter Verwendung verschiedener Cloudservices aufzeigt.

Um die gestellte Aufgabe zu lösen, wurden in der Anforderungsanalyse die Voraussetzungen für das System dargelegt.
Dabei wurde das grobe Konzept des Prototypen erklärt und die unterschiedlichen Anforderungen an die Cloudservices und die zu verwendenden Technologien erarbeitet.
Weiterhin dienten die erstellten Use-Cases dem besseren Verständnis der Funktionsweise des Prototypen.
Die Evaluation bestehender Anwendung und Dienste zeigt Umsetzungsmöglichkeiten und zugleich Abgrenzungskriterien für CloudGrid auf.

Anschließend wurde im Kapitel Systementwurf konkreter auf die geplante Umsetzung des Prototypen eingegangen.
Dabei wurde sowohl die zu verwendende Programmierumgebung ermittelt, als auch die Architektur des Systems aufgezeigt.
Diese konnte in drei Schichten, Datenhaltungs-, Anwendungslogik- und Präsentationsschicht, unterteilt werden.
Die Evaluation der Cloudservices wertete, sowohl aus technischer, als auch aus rechtlicher Sicht, bestehende Dienste, anhand der zuvor definierten Kriterien, aus.

Die konkrete Umsetzung des Prototypen wurde daraufhin im Kapitel Implementierung ausgearbeitet.
Dazu wurde neben der Projektstruktur, die Funktionsweise externer und auch selbst entwickelter Module erklärt und deren Einbindung in das System erläutert.
Weiterhin wurden die Realisierung der einzelnen Schichten erörtert.

Abschließend wurde im Abschnitt Evaluation und Demonstration der entwickelte Prototyp bewertet und die Handhabung, sowie die Funktionsweise, aufgezeigt.

Im Prototypen von CloudGrid wurden alle zuvor erarbeiteten Anforderungen umgesetzt.
Bei einer zukünftige Weiterentwicklung des vorgestellten Systems sollte primär auf eine Erweiterung des Cloudservice Portfolios gesetzt werden.
Um dies zu realisieren, sollten sowohl mehr Authentifizierungsmethoden unterstützt werden, als auch das Datenformat \ac{XML}, da dies von mehreren Anbietern verwendet wird.
Dadurch könnten bereits vier weitere, der in Abschnitt \ref{systementwurf-cloudanbieter} evaluierten Anbieter, eingebunden werden.
Darüber hinaus würde die Anzahl nochmals steigen, wenn auch kostenpflichtige Anbieter hinzugefügt werden.
Jedoch bleibt der Nachteil bestehen, dass anbieterspezifische Funktion nicht in CloudGrid integrierbar sind, wie beispielsweise die Versionierung von Dropbox.

Weiterhin sollte eine Synchronisierung über mehrere Clients realisiert werden.
Das erhöht die Nutzbarkeit für den Anwender und entspricht der Funktionalität aktueller Clientanwendungen der Anbieter.
Hierbei müssen wahrscheinlich größere systemarchitektonische Anpassungen vorgenommen werden.
Ein Konzept wäre, die lokalen Benutzerdaten von CloudGrid ebenfalls redundant bei den Cloudservices vorzuhalten, um diese auf einem weiteren Client einzubinden.
Ebenfalls denkbar wäre die Einbindung eines weiteren Serverdienstes, der die entsprechenden Informationen vorhält.
Allerdings ist diese Möglichkeit abweichend von der momentanen Grundidee von CloudGrid, dem Benutzer ein System zu ermöglichen, bei dem er jederzeit Überblick über den Verbleib seiner Daten hat.

Zudem würde die Umsetzung mobiler Anwendungen, für Smartphones und Tablets, die Anwenderfreundlichkeit erhöhen.
Auch hier müssten konzeptionelle Änderungen durchgeführt werden, welche sich möglicherweise mit denen des Multiclient Konzepts gleichen.

Auf Seiten der Node.js Anwendung würde die Erweiterung des \frqq watchr\flqq\ Moduls um die Events \frqq Umbenennen\flqq\ und \frqq Verschieben\flqq\ die Performance steigern und unnötige Dateioperationen vermeiden.
Weiterhin ist es möglich, weitere Einstellungsmöglichkeiten, wie das Setzen eines anderen oder eines weiteren zu überprüfenden Ordners oder auch die Wahl des Verschlüsselungsalgorithmus.
Dabei muss jedoch beachtet werden, dass sich solch eine Option auf bereits hochgeladene Dateien auswirken würde.

Wie bereits in Abschnitt \ref{evaluation-des-system} beschrieben, sollte auch die Einbindung weiterer Cloudservices modularer und generischer gestaltet werden.
Momentan müssen mehrere Dateien bearbeitet werden, um einen Dienst zu CloudGrid hinzuzufügen.
Besser wäre es, wenn es einen Ordner geben würde, wo Module für Cloudservices vorgehalten und weitere hinzugefügt werden können.
Dieser wird beim Systemstart auf neue Module geprüft und diese entsprechend eingebunden.

Abschließend sollte die \ac{GUI} erweitert werden, um die Benutzerfreundlichkeit zu steigern.
Dazu zählen mehr Informationen für den Benutzer, wie beispielsweise ein Hinweis, wenn der Speicherplatz eines Anbieters ausgereizt ist oder ein Anbieter nicht mehr verfügbar ist.

Letztendlich zeigt das hier vorgestellte System, erfolgreich und in einer prototypischen Qualität, die Funktionsweise des erarbeiteten Konzeptes auf.
Das Ergebnis kann somit als solide Grundlage für eine Weiterentwicklung angesehen werden.