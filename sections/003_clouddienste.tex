\chapter{Grundlagen}

\section{Clouddienste}
Das Thema Cloud-Computing ist momentan allgegenwärtig in der Informatik.
Dennoch gibt es keine standardisierte Definition, was die "`Cloud"' ist.
Das Wort "`Cloud"' oder im deutschen "`Wolke"' weist jedoch auf zwei wichtige Konzepte hin, nämlich Virtualisierung und Skalierbarkeit\cite[vgl.][Seite 35]{hoell11}.
Bei der Virtualisierung werden Computerressourcen transparent zusammengefasst beziehungsweise aufgeteilt.
Dadurch wird eine beliebige Sicht auf die Infrastruktur geschaffen, wodurch keine systembedingten Abhängigkeiten für eine Anwendung entstehen\cite[vgl.][Seite 2]{baun10}.

Der zweite wichtige Aspekt, die Skalierbarkeit, ermöglicht es, zusätzliche Ressourcen ohne großen Aufwand in das System zu integrieren, sei es um Hardware nachzurüsten oder bestehende Systeme zusammen zu fassen.
Das hat den Vorteil, dass ein Unternehmen nicht mehr in diese investieren und selbst verwalten muss.
Es kann sich auf die Umsetzung seiner Geschäftsidee konzentrieren und bei wachsenden Anforderungen Ressourcen flexibel vom Provider beziehen\cite[vgl.][Seite 2]{baun10}.

Als großer Kritikpunkt lassen sich hingegen die Sicherheitsbedenken aufführen.
Dadurch, dass die Anwendungslogik an eine externe Firma ausgelagert wird, verliert das Unternehmen die Kontrolle über seine Daten.
Zudem muss es sich auf die Sicherheitsmaßnahmen der Anbieter verlassen und hat keinen Einfluss auf selbige.
Besonders bei sensiblen Daten, wie Krankenakten von Patienten oder Kundendaten einer Bank, spielen Sicherheitsaspekte eine übergeordnete Rolle, wodurch unter Umständen von der Verwendung von Cloud-Computing abgeraten werden muss.
Auch die Verfügbarkeit und Performance von Ressourcen in einer Cloud geben Grund zur Kritik.
Sollte ein Service für mehrere Stunden nicht verfügbar sein, so kann ein erheblicher wirtschaftlicher Schaden für die betroffene Firma entstehen.
Zudem existieren momentan keine Standards bei der Implementierung von Ressourcen, womit ein schneller Wechsel des Anbieters in der Regel nicht einfach möglich ist\cite[vgl.][Seite 3]{baun10}.

Je nach Anwendungsfall muss ein Unternehmen sich daher detailliert überlegen, ob es Cloud-Computing verwenden möchte und welche Form und welcher Anbieter dabei seinen Ansprüchen genügt.

Neben der Auslagerung von Anwendungslogiken gibt es mehrere Dienste die vorgefertigte Lösungen anbieten.
So streamen Spotify\footnote{\url{http://www.spotify.com}} oder Deezer\footnote{\url{http://www.deezer.com}} Musik, Youtube\footnote{\url{http://www.youtube.de}} streamt Videos, Google Docs\footnote{\url{https://docs.google.com/}} bietet eine komplette Office Suite zum kollaborativen Arbeiten an und sogar Bildbearbeitung ist mit Diensten wie dem Photoshop Express Editor\footnote{\url{http://www.photoshop.com/tools/expresseditor}} möglich.
Die Vorteile für den Benutzer liegen dabei auf der Hand.
Er kann mobil und über verschiedene Geräte hinweg seine Daten abrufen und bearbeiten, ohne dabei auf zusätzliche Hardware wie USB-Sticks oder DVDs zurückgreifen zu müssen.
Darüber hinaus muss sich der Anwender nicht mehr manuell um die Aktualisierung seiner Daten auf verschiedene Geräte kümmern.
Beispielsweise bearbeitet er ein Foto in der Cloud, speichert dieses dort und kann es später von einem anderen Rechner herunterladen.
Dadurch, dass sich der Anbieter des Clouddienstes um ein sinnvolles Datenbackup kümmern muss, beispielsweise durch redundante Datenhaltung auf mehreren Servern, wird für den Benutzer eine Ausfallsicherheit gewährleistet.
Sollte der Rechner defekt sein, gehen die Daten nicht verloren, sondern sind weiterhin auf dem Server verfügbar.

Zudem kann die Architektur von Clouddiensten aus zwei Perspektiven betrachtet werden.
Einerseits aus organisatorischer Sicht und andererseits aus technischer Sicht\cite[vgl.][Seite 25]{baun10}.
Zu den erstgenannten zählen die "`Private Cloud"', die "`Public Cloud"' und die "`Hybrid Cloud"'\cite[vgl.][Seite 25]{baun10}.
Clouddienste, die aus technischen Sicht betrachtet werden, können in "`\ac{IaaS}"', "`\ac{PaaS}"' und "`\ac{SaaS} eingeteilt werden\cite[vgl.][Seite 25]{baun10}.

\subsection{Clouddienste aus organisatorischer Sicht}
Bei der organisatorischen Betrachtung von Clouddiensten, werden Benutzer und Anbieter strukturell voneinander getrennt.
Der Benutzer verwendet dabei einen Dienst, welcher vom Anbieter offeriert wird.
Beide Einheiten können dem selben Unternehmen angehören, werden jedoch separate betrachtet.

\subsubsection{Private Cloud}
Unter "`Private Cloud"' versteht man eine Cloud, bei der sowohl die Benutzer als auch der Anbieter derselben organisatorischen Einheit angehören\cite[vgl.][Seite 25 f.]{baun10}.
Es ist also ein geschlossenes Netzwerk, wie beispielsweise ein firmeninternes Intranet, welches den Benutzern exklusive Ressourcen zur Verfügung stellt\cite[vgl.][Seite 40]{hoell11}.
Hauptgrund für die Verwendung von "`Private Clouds"' sind Sicherheitsaspekte.
Gerade bei sensiblen Daten ist diese Art der Cloud von großem Vorteil, da sich dadurch viele Sicherheitsprobleme beseitigen lassen.
So bleibt die Kontrolle über die Daten bei der Organisation selbst und wird nicht an Dritte weitergeben\cite[vgl.][Seite 26]{baun10}.
Außerdem können einzelne Ressourcen kontrollierter an Benutzer weitergegeben werden.
Ein Mitarbeiter erhält so zwar Zugang zu seinem eigenen Dienstplan, jedoch nicht zu dem seiner Kollegen, wohingegen der Geschäftsführer Zugriff auf alle Dienstpläne hat.
Auch Hardwareressourcen können flexibel skaliert und entsprechend der aktuellen Situation angepasst werden.

Jedoch muss die Verwaltung und Wartung der Cloud, sowohl soft- als auch hardwareseitig, von der Organisation selbst übernommen werden.
Dieser Aspekt ist aus unternehmerischer Sicht kostenintensiver und daher möglicherweise unattraktiver.
Besonders da diese, je nach Umfang, von einem oder mehreren Administratoren gepflegt werden muss.

\subsubsection{Public Cloud}
Eine "`Public Cloud"' kann als ein am Markt angebotener Clouddienste angesehen werden\cite[vgl.][Seite 38]{hoell11}.
Es sind also Cloud-Angebote, bei denen der Anbieter und die potenziellen Benutzer nicht zu ein- und derselben organisatorischen Einheit gezählt werden\cite[vgl.][Seite 25]{baun10}.
Meist sind jene Anbieter gemeint, wenn von der "`Cloud"' gesprochen wird.
Die Ressourcen solcher Dienste werden meist von vielen Nutzern gleichzeitig geteilt, welche nur durch die Virtualisierung als eigene Umgebung für den einzelnen Benutzer erscheinen.
Viele Dienstleister bieten darüber hinaus ein Web-Portal oder gar eine \ac{API} zum Interagieren mit dem Dienst an.
Diese sind für den Betreiber leicht skalierbar, sodass weitere Hardwareressourcen jederzeit eingebunden werden können.
Besonders bei wachsenden Anforderungen an das System ist dieser Aspekt essenziell.
Auf Seiten des Nutzers lässt sich dieses Prinzip nur schwer anwenden.
"`Eine Anpassung an spezifische Anwenderanforderungen ist üblicherweise nur sehr eingeschränkt möglich/erwünscht bzw. steht den Skalierungs- und Effizienzinteressen des Betreibers gegenüber"'\cite[Seite 39]{hoell11}.

Eine übliche Bezahlmethode ist "'Pay-per-Use"', das bedeutet je nach benutzten Ressourcen fallen mehr oder weniger hohe Kosten an, sowie die "`Flatrate"', für die der Nutzer einen festen Betrag bezahlt, wobei er einen vordefinierten Umfang der Ressourcen zur Verfügung gestellt bekommt.
Ein prominenter Vertreter für "`Pay-per-User"' ist Amazon EC2\footnote{\url{http://aws.amazon.com/de/ec2}}, wohingegen beispielsweise Dropbox\footnote{\url{https://www.dropbox.com/business}} oder Spotifiy\footnote{\url{https://www.spotify.com/de}} auf ein Flaterate-Model setzen.

Gerade im Gegensatz zu "`Private Clouds"' ergeben sich bei den "`Public Clouds"' zahlreiche Bedenken.
So muss für jedes System "`hinsichtlich Compliance, Sicherheit, Verfügbarkeit, aber auch Performance"'\cite[Seite 40]{hoell11}. detailliert geprüft werden, ob eine "`Public Cloud"' den Anforderungen gerecht wird.
Durch den Verlust der, im Vergleich zu den "`Private Clouds"', absoluten Kontrolle über die Daten des Benutzers, müssen beim Verstoß gegen vertraglich vereinbarte Regelungen, diese erst juristisch durchgesetzt werden\cite[vgl.][Seite 40]{hoell11}.
Besonders der zeitliche Aspekt von Gerichtsverfahren birgt ein großes Risiko für Unternehmen.

\subsubsection{Hybrid Cloud}
"`Hybrid Clouds"' sind eine Mischform aus "`Public Clouds"' und "`Private Clouds"'.
Hierbei wird eine "`Private Cloud"' auf organisationseigenen Ressourcen betrieben, jedoch werden zusätzlich Services herkömmlicher Datenverarbeitungslösungen mittels \acp{API} von "`Public Cloud"' Anbietern eingebunden.
Somit können konkrete Anforderungen an ein System für das jeweilige Unternehmen angepasst werden.
Etwaige Sicherheitsvorgaben können eingehalten werden und gleichzeitig Skalierungseffekte genutzt werden\cite[vgl.][Seite 42]{hoell11}.
Durch Virtualisierung wird auch bei "`Hybrid Clouds"' dem Benutzer ein geschlossenes System dargestellt, sodass dieser weder einen Unterschied noch einen Übergang zwischen den einzelnen Clouds bemerkt\cite[vgl.][Seite 11]{hoell11}.

Als Nachteil kann eine erhöhte Komplexität bei der Integration solcher Systeme aufgeführt werden.
Eine nahtlose und für den Benutzer transparente Implementierung muss sorgfältig geplant und von den entsprechenden Administratoren und Entwicklern eingebunden werden.
Zudem muss detailliert geprüft werden, welche Komponenten in die "`Public Clouds"' auszulagern sind, damit die Nachteile selbiger nicht das Gesamtsystem beeinträchtigen.

\subsection{Clouddienste aus technischer Sicht}
Bei der technischen Betrachtung der Clouddienste wird eine abstrahiert Sicht auf Hard- und Software geboten.
Der Anbieter kümmert sich um die Wartung und Pflege der Ressourcen, wohingegen der Kunde virtualisierte Ressourcen angeboten bekommt.

\subsubsection{Infrastructure as a Service}
Einem Benutzer wird bei der Verwendung der \ac{IaaS}-Schicht eine abstrahierte Sicht auf die Hardware, also unter anderem auf ein bestehendes Netzwerk, einen Server und deren räumliche und klimatische Bedingungen oder auch nur auf Massenspeichermedien, geboten\cite[vgl.][Seite 47]{hoell11}.
Dabei wird dem Kunden meist eine Teilmenge der eigentlichen Ressourcen zur Verfügung gestellt.
"`Man könnte IaaS auch als eine im Wesentlichen durch Einsatz von Software abstrahierte und virtualisierte Rechenzentrumsressource auffassen"'\cite[Seite 47]{hoell11}.
Das bedeutet, dass sich meist mehrere Kunden ein und dieselbe Ressource teilen, welche lediglich auf Softwareebene virtualisiert wird.
Besonders in Hinsicht auf die Skalierbarkeit der Ressourcen ist das ein erheblicher Vorteil.

Als Beispiel für solche Services kann man virtuelle Server von Hostingprovidern oder auch Dienste wie Amazon EC2 und Dropbox nennen.
Letztere bieten darüber hinaus noch Schnittstellen zur Kommunikation mit den Diensten an, wohin gegen virtuelle Server meist über einen \ac{SSH} Zugang verfügen.

\subsubsection{Platform as a Service}
Die \ac{PaaS}-Schicht baut auf der \ac{IaaS}-Schicht auf und bietet dem Benutzer Anwendungen, Datenbanken oder Webservices, mit denen er arbeiten kann.
Oftmals wird diese auch als \frqq Middleware\flqq\ bezeichnet\cite[vgl.][Seite 47]{hoell11}.
In der Praxis richtet sich dieser Service meist an Entwickler.
Der Vorteil gegenüber \ac{IaaS} ist ein verminderter Wartungsaufwand.
Das bedeutet, dass es nicht mehr notwendig ist Software auf einem virtuellen Server zu aktualisieren oder eine Firewall auf selbigem zu konfigurieren.
Ähnlich wie bei \ac{IaaS} ist auch dieser Service leicht skalierbar und kann flexibel an die Bedürfnisse eines Unternehmens angepasst werden.

Prominente Vertreter für diese Art von Services sind Microsoft Azure\footnote{\url{http://www.windowsazure.com/de-de}} und Google App Engine\footnote{\url{https://cloud.google.com/products}}.
Beide verfügen über ein ähnliches Angebot und ermöglichen eine stundenweise Nutzung und Bezahlung.

\subsubsection{Software as a Service}
Zur \ac{SaaS}-Schicht gehören Anwendungen in der Cloud, die direkt an den Endkunden adressiert sind\cite[vgl.][Seite 35]{baun10}.
Dabei wird die Software direkt angeboten, ohne dass sich der Kunde um die Wartung weiterer Ressourcen kümmern muss.
Alle zur Ausführung erforderlichen Ressourcen werden durch den Anbieter gestellt und gepflegt.
"`Den Anspruch als Cloud-Computing-Service erfüllt es allerdings nur dann, wenn das Angebot massiv skalierbar, mehrmandantenfähig und elastisch flexibel ist"'\cite[Seite 47]{hoell11}.

Diese Art von Service ist die derzeit am häufigsten angebotene und diskutierte Form Cloudservices\cite[vgl.][Seite 47]{hoell11}.
Beispiele für diese Art des Services sind Google Maps, Google Docs oder das \ac{CRM} von salesforce\footnote{\url{https://www.salesforce.com/de/form/sem/landing/sales-cloud.jsp}}.
%
%\subsection{Zusammenfassung}
%Cloudservices erfreuen sich großer Beliebtheit.
%Aktuell ist diese ein häufig diskutiertes Thema und immer mehr Anbieter präsentieren sich auf dem Markt.
%Jedoch muss ein Unternehmen genau abwägen welche Services es nutzen möchte und welche optimal zu dem eigenen Produkt passen.
%\ac{SaaS}-Services richten sich dabei oftmals an Endverbraucher und sind für Firmen ehr uninteressant, jedoch können diese gerade beim kollaborativen Arbeiten einen großen Mehrwert bieten.
%In der Praxis erweisen sich \ac{PaaS} und \ac{IaaS} Services als sinnvolle Lösungen für einzelne Unternehmungen.
%Gerade durch eine preiswerte Nutzung und attraktiven Bezahlmodellen wie Pay-per-use sind "`Public Clouds"' oftmals eine gute Wahl.
%Jedoch sollten für jeden Anwendungsfall detailliert die Vor- und Nachteile abgewogen werden.
%Besonders im Bereich Datensicherheit.
%Unternehmen welche mit sensiblen Daten arbeiten, sollten daher lieber in eine "`Private Cloud"' investieren oder, insofern es zum Konzept des Produktes passt, in eine "`Hybrid Cloud"', welche Teilaskpekte in einen "`Public Cloud"' Service auslagert.
%